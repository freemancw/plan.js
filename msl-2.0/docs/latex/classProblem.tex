\section{Problem  Class Reference}
\label{classProblem}\index{Problem@{Problem}}
An interface class that provides the primary input to a planner. 


{\tt \#include $<$problem.h$>$}

\subsection*{Public Methods}
\begin{CompactItemize}
\item 
{\bf Problem} ({\bf Geom} $\ast$geom, {\bf Model} $\ast$model, string path)
\begin{CompactList}\small\item\em Problem must be given any instance of {\bf Geom} {\rm (p.\,\pageref{classGeom})} and any instance of {\bf Model} {\rm (p.\,\pageref{classModel})} from each of their class hierarchies.\item\end{CompactList}\item 
virtual {\bf $\sim$Problem} ()
\begin{CompactList}\small\item\em Empty destructor.\item\end{CompactList}\item 
void {\bf Set\-Geom} ({\bf Geom} $\ast$geom)
\begin{CompactList}\small\item\em Change the instance of {\bf Geom} {\rm (p.\,\pageref{classGeom})}.\item\end{CompactList}\item 
void {\bf Set\-Model} ({\bf Model} $\ast$model)
\begin{CompactList}\small\item\em Change the instance of {\bf Model} {\rm (p.\,\pageref{classModel})}.\item\end{CompactList}\item 
virtual list$<$ {\bf MSLVector} $>$ {\bf Get\-Inputs} (const {\bf MSLVector} \&{\bf x})
\begin{CompactList}\small\item\em Return a list of possible inputs, which may depend on state.\item\end{CompactList}\item 
virtual list$<$ {\bf MSLVector} $>$ {\bf Get\-Inputs} ()
\begin{CompactList}\small\item\em Return a list of possible inputs.\item\end{CompactList}\item 
virtual {\bf MSLVector} {\bf Integrate} (const {\bf MSLVector} \&{\bf x}, const {\bf MSLVector} \&u, const double \&deltat)
\begin{CompactList}\small\item\em Perform integration from state x, using input u, over time deltat.\item\end{CompactList}\item 
virtual {\bf MSLVector} {\bf Interpolate\-State} (const {\bf MSLVector} \&x1, const {\bf MSLVector} \&x2, const double \&a)
\begin{CompactList}\small\item\em Linearly interpolate two states while respecting topology.\item\end{CompactList}\item 
virtual double {\bf Metric} (const {\bf MSLVector} \&x1, const {\bf MSLVector} \&x2)
\begin{CompactList}\small\item\em A distance metric defined in {\bf Model} {\rm (p.\,\pageref{classModel})}.\item\end{CompactList}\item 
virtual {\bf MSLVector} {\bf State\-To\-Configuration} (const {\bf MSLVector} \&{\bf x})
\begin{CompactList}\small\item\em A method that converts a {\bf Model} {\rm (p.\,\pageref{classModel})} state in to a {\bf Geom} {\rm (p.\,\pageref{classGeom})} configuration.\item\end{CompactList}\item 
virtual {\bf MSLVector} {\bf State\-Difference} (const {\bf MSLVector} \&x1, const {\bf MSLVector} \&x2)
\begin{CompactList}\small\item\em Compute a {\bf MSLVector} {\rm (p.\,\pageref{classMSLVector})} based on x2-x1. In R$^\wedge$n, the states are simply subtracted to make the {\bf MSLVector} {\rm (p.\,\pageref{classMSLVector})}. This method exists to make things work correctly for other state-space topologies.\item\end{CompactList}\item 
virtual bool {\bf Satisfied} (const {\bf MSLVector} \&{\bf x})
\begin{CompactList}\small\item\em This takes the logical AND of Collision\-Free from {\bf Geom} {\rm (p.\,\pageref{classGeom})}, and Satisfied from {\bf Model} {\rm (p.\,\pageref{classModel})}.\item\end{CompactList}\item 
virtual bool {\bf Collision\-Free} (const {\bf MSLVector} \&q)
\begin{CompactList}\small\item\em The collision checker passed in from {\bf Geom} {\rm (p.\,\pageref{classGeom})}.\item\end{CompactList}\item 
virtual double {\bf Distance\-Comp} (const {\bf MSLVector} \&q)
\begin{CompactList}\small\item\em The distance computation algorithm from {\bf Geom} {\rm (p.\,\pageref{classGeom})}.\item\end{CompactList}\item 
virtual {\bf MSLVector} {\bf Configuration\-Difference} (const {\bf MSLVector} \&q1, const {\bf MSLVector} \&q2)
\begin{CompactList}\small\item\em Compute a {\bf MSLVector} {\rm (p.\,\pageref{classMSLVector})} based on q2-q1. In R$^\wedge$n, the configurations are simply subtracted to make the {\bf MSLVector} {\rm (p.\,\pageref{classMSLVector})}. This method exists to make things work correctly for other configuration-space topologies.\item\end{CompactList}\end{CompactItemize}
\subsection*{Public Attributes}
\begin{CompactItemize}
\item 
string {\bf File\-Path}
\begin{CompactList}\small\item\em The directory in which all files for a problem will be stored.\item\end{CompactList}\item 
int {\bf Num\-Bodies}
\begin{CompactList}\small\item\em The number of rigid bodies.\item\end{CompactList}\item 
int {\bf State\-Dim}
\begin{CompactList}\small\item\em The dimenson of the state space.\item\end{CompactList}\item 
int {\bf Input\-Dim}
\begin{CompactList}\small\item\em The dimension of the input space.\item\end{CompactList}\item 
int {\bf Geom\-Dim}
\begin{CompactList}\small\item\em The dimension of the geometric model.\item\end{CompactList}\item 
{\bf MSLVector} {\bf Lower\-State}
\begin{CompactList}\small\item\em {\bf MSLVector} {\rm (p.\,\pageref{classMSLVector})} of minimum values for each state variable.\item\end{CompactList}\item 
{\bf MSLVector} {\bf Upper\-State}
\begin{CompactList}\small\item\em {\bf MSLVector} {\rm (p.\,\pageref{classMSLVector})} of maximum values for each state variable.\item\end{CompactList}\item 
{\bf MSLVector} {\bf Initial\-State}
\begin{CompactList}\small\item\em The starting state for a planner.\item\end{CompactList}\item 
{\bf MSLVector} {\bf Goal\-State}
\begin{CompactList}\small\item\em The goal state for a planner.\item\end{CompactList}\item 
{\bf MSLVector} {\bf Max\-Deviates}
\begin{CompactList}\small\item\em Maximum displacement of geometry with respect to change in each variable.\item\end{CompactList}\end{CompactItemize}
\subsection*{Protected Attributes}
\begin{CompactItemize}
\item 
{\bf Geom} $\ast$ {\bf G}
\begin{CompactList}\small\item\em Need to define a geometry for collision detection.\item\end{CompactList}\item 
{\bf Model} $\ast$ {\bf M}
\begin{CompactList}\small\item\em xdot = f(x,u), integration technique, state bounds.\item\end{CompactList}\end{CompactItemize}


\subsection{Detailed Description}
An interface class that provides the primary input to a planner.

This interface class contains protected instances of {\bf Geom} {\rm (p.\,\pageref{classGeom})} and {\bf Model} {\rm (p.\,\pageref{classModel})}. Wrappers to methods from {\bf Geom} {\rm (p.\,\pageref{classGeom})} provide collision detection and distance computation. Wrappers to methods from {\bf Model} {\rm (p.\,\pageref{classModel})} provide incremental simulation of a kinematic or dynamical system. It is expected a planner can get all (or nearly all) of the information it needs from Problem. 



\subsection{Constructor \& Destructor Documentation}
\index{Problem@{Problem}!Problem@{Problem}}
\index{Problem@{Problem}!Problem@{Problem}}
\subsubsection{\setlength{\rightskip}{0pt plus 5cm}Problem::Problem ({\bf Geom} $\ast$ {\em geom}, {\bf Model} $\ast$ {\em model}, string {\em path})}\label{classProblem_a0}


Problem must be given any instance of {\bf Geom} {\rm (p.\,\pageref{classGeom})} and any instance of {\bf Model} {\rm (p.\,\pageref{classModel})} from each of their class hierarchies.

\index{Problem@{Problem}!~Problem@{$\sim$Problem}}
\index{~Problem@{$\sim$Problem}!Problem@{Problem}}
\subsubsection{\setlength{\rightskip}{0pt plus 5cm}virtual Problem::$\sim$Problem ()\hspace{0.3cm}{\tt  [inline, virtual]}}\label{classProblem_a1}


Empty destructor.



\subsection{Member Function Documentation}
\index{Problem@{Problem}!CollisionFree@{CollisionFree}}
\index{CollisionFree@{CollisionFree}!Problem@{Problem}}
\subsubsection{\setlength{\rightskip}{0pt plus 5cm}bool Problem::Collision\-Free (const {\bf MSLVector} \& {\em q})\hspace{0.3cm}{\tt  [virtual]}}\label{classProblem_a12}


The collision checker passed in from {\bf Geom} {\rm (p.\,\pageref{classGeom})}.

\index{Problem@{Problem}!ConfigurationDifference@{ConfigurationDifference}}
\index{ConfigurationDifference@{ConfigurationDifference}!Problem@{Problem}}
\subsubsection{\setlength{\rightskip}{0pt plus 5cm}{\bf MSLVector} Problem::Configuration\-Difference (const {\bf MSLVector} \& {\em q1}, const {\bf MSLVector} \& {\em q2})\hspace{0.3cm}{\tt  [virtual]}}\label{classProblem_a14}


Compute a {\bf MSLVector} {\rm (p.\,\pageref{classMSLVector})} based on q2-q1. In R$^\wedge$n, the configurations are simply subtracted to make the {\bf MSLVector} {\rm (p.\,\pageref{classMSLVector})}. This method exists to make things work correctly for other configuration-space topologies.

\index{Problem@{Problem}!DistanceComp@{DistanceComp}}
\index{DistanceComp@{DistanceComp}!Problem@{Problem}}
\subsubsection{\setlength{\rightskip}{0pt plus 5cm}double Problem::Distance\-Comp (const {\bf MSLVector} \& {\em q})\hspace{0.3cm}{\tt  [virtual]}}\label{classProblem_a13}


The distance computation algorithm from {\bf Geom} {\rm (p.\,\pageref{classGeom})}.

\index{Problem@{Problem}!GetInputs@{GetInputs}}
\index{GetInputs@{GetInputs}!Problem@{Problem}}
\subsubsection{\setlength{\rightskip}{0pt plus 5cm}list$<$ {\bf MSLVector} $>$ Problem::Get\-Inputs ()\hspace{0.3cm}{\tt  [virtual]}}\label{classProblem_a5}


Return a list of possible inputs.

\index{Problem@{Problem}!GetInputs@{GetInputs}}
\index{GetInputs@{GetInputs}!Problem@{Problem}}
\subsubsection{\setlength{\rightskip}{0pt plus 5cm}list$<$ {\bf MSLVector} $>$ Problem::Get\-Inputs (const {\bf MSLVector} \& {\em x})\hspace{0.3cm}{\tt  [virtual]}}\label{classProblem_a4}


Return a list of possible inputs, which may depend on state.

\index{Problem@{Problem}!Integrate@{Integrate}}
\index{Integrate@{Integrate}!Problem@{Problem}}
\subsubsection{\setlength{\rightskip}{0pt plus 5cm}{\bf MSLVector} Problem::Integrate (const {\bf MSLVector} \& {\em x}, const {\bf MSLVector} \& {\em u}, const double \& {\em deltat})\hspace{0.3cm}{\tt  [virtual]}}\label{classProblem_a6}


Perform integration from state x, using input u, over time deltat.

\index{Problem@{Problem}!InterpolateState@{InterpolateState}}
\index{InterpolateState@{InterpolateState}!Problem@{Problem}}
\subsubsection{\setlength{\rightskip}{0pt plus 5cm}{\bf MSLVector} Problem::Interpolate\-State (const {\bf MSLVector} \& {\em x1}, const {\bf MSLVector} \& {\em x2}, const double \& {\em a})\hspace{0.3cm}{\tt  [virtual]}}\label{classProblem_a7}


Linearly interpolate two states while respecting topology.

If a=0, then x1 is returned; if a=1, then x2 is returned. All intermediate values of \$a $\backslash$in [0,1]\$ yield intermediate states. This method is defined by {\bf Model} {\rm (p.\,\pageref{classModel})}. \index{Problem@{Problem}!Metric@{Metric}}
\index{Metric@{Metric}!Problem@{Problem}}
\subsubsection{\setlength{\rightskip}{0pt plus 5cm}double Problem::Metric (const {\bf MSLVector} \& {\em x1}, const {\bf MSLVector} \& {\em x2})\hspace{0.3cm}{\tt  [virtual]}}\label{classProblem_a8}


A distance metric defined in {\bf Model} {\rm (p.\,\pageref{classModel})}.

\index{Problem@{Problem}!Satisfied@{Satisfied}}
\index{Satisfied@{Satisfied}!Problem@{Problem}}
\subsubsection{\setlength{\rightskip}{0pt plus 5cm}bool Problem::Satisfied (const {\bf MSLVector} \& {\em x})\hspace{0.3cm}{\tt  [virtual]}}\label{classProblem_a11}


This takes the logical AND of Collision\-Free from {\bf Geom} {\rm (p.\,\pageref{classGeom})}, and Satisfied from {\bf Model} {\rm (p.\,\pageref{classModel})}.

\index{Problem@{Problem}!SetGeom@{SetGeom}}
\index{SetGeom@{SetGeom}!Problem@{Problem}}
\subsubsection{\setlength{\rightskip}{0pt plus 5cm}void Problem::Set\-Geom ({\bf Geom} $\ast$ {\em geom})}\label{classProblem_a2}


Change the instance of {\bf Geom} {\rm (p.\,\pageref{classGeom})}.

\index{Problem@{Problem}!SetModel@{SetModel}}
\index{SetModel@{SetModel}!Problem@{Problem}}
\subsubsection{\setlength{\rightskip}{0pt plus 5cm}void Problem::Set\-Model ({\bf Model} $\ast$ {\em model})}\label{classProblem_a3}


Change the instance of {\bf Model} {\rm (p.\,\pageref{classModel})}.

\index{Problem@{Problem}!StateDifference@{StateDifference}}
\index{StateDifference@{StateDifference}!Problem@{Problem}}
\subsubsection{\setlength{\rightskip}{0pt plus 5cm}{\bf MSLVector} Problem::State\-Difference (const {\bf MSLVector} \& {\em x1}, const {\bf MSLVector} \& {\em x2})\hspace{0.3cm}{\tt  [virtual]}}\label{classProblem_a10}


Compute a {\bf MSLVector} {\rm (p.\,\pageref{classMSLVector})} based on x2-x1. In R$^\wedge$n, the states are simply subtracted to make the {\bf MSLVector} {\rm (p.\,\pageref{classMSLVector})}. This method exists to make things work correctly for other state-space topologies.

\index{Problem@{Problem}!StateToConfiguration@{StateToConfiguration}}
\index{StateToConfiguration@{StateToConfiguration}!Problem@{Problem}}
\subsubsection{\setlength{\rightskip}{0pt plus 5cm}{\bf MSLVector} Problem::State\-To\-Configuration (const {\bf MSLVector} \& {\em x})\hspace{0.3cm}{\tt  [virtual]}}\label{classProblem_a9}


A method that converts a {\bf Model} {\rm (p.\,\pageref{classModel})} state in to a {\bf Geom} {\rm (p.\,\pageref{classGeom})} configuration.



\subsection{Member Data Documentation}
\index{Problem@{Problem}!FilePath@{FilePath}}
\index{FilePath@{FilePath}!Problem@{Problem}}
\subsubsection{\setlength{\rightskip}{0pt plus 5cm}string Problem::File\-Path}\label{classProblem_m0}


The directory in which all files for a problem will be stored.

\index{Problem@{Problem}!G@{G}}
\index{G@{G}!Problem@{Problem}}
\subsubsection{\setlength{\rightskip}{0pt plus 5cm}{\bf Geom}$\ast$ Problem::G\hspace{0.3cm}{\tt  [protected]}}\label{classProblem_n0}


Need to define a geometry for collision detection.

\index{Problem@{Problem}!GeomDim@{GeomDim}}
\index{GeomDim@{GeomDim}!Problem@{Problem}}
\subsubsection{\setlength{\rightskip}{0pt plus 5cm}int Problem::Geom\-Dim}\label{classProblem_m4}


The dimension of the geometric model.

\index{Problem@{Problem}!GoalState@{GoalState}}
\index{GoalState@{GoalState}!Problem@{Problem}}
\subsubsection{\setlength{\rightskip}{0pt plus 5cm}{\bf MSLVector} Problem::Goal\-State}\label{classProblem_m8}


The goal state for a planner.

\index{Problem@{Problem}!InitialState@{InitialState}}
\index{InitialState@{InitialState}!Problem@{Problem}}
\subsubsection{\setlength{\rightskip}{0pt plus 5cm}{\bf MSLVector} Problem::Initial\-State}\label{classProblem_m7}


The starting state for a planner.

\index{Problem@{Problem}!InputDim@{InputDim}}
\index{InputDim@{InputDim}!Problem@{Problem}}
\subsubsection{\setlength{\rightskip}{0pt plus 5cm}int Problem::Input\-Dim}\label{classProblem_m3}


The dimension of the input space.

\index{Problem@{Problem}!LowerState@{LowerState}}
\index{LowerState@{LowerState}!Problem@{Problem}}
\subsubsection{\setlength{\rightskip}{0pt plus 5cm}{\bf MSLVector} Problem::Lower\-State}\label{classProblem_m5}


{\bf MSLVector} {\rm (p.\,\pageref{classMSLVector})} of minimum values for each state variable.

\index{Problem@{Problem}!M@{M}}
\index{M@{M}!Problem@{Problem}}
\subsubsection{\setlength{\rightskip}{0pt plus 5cm}{\bf Model}$\ast$ Problem::M\hspace{0.3cm}{\tt  [protected]}}\label{classProblem_n1}


xdot = f(x,u), integration technique, state bounds.

\index{Problem@{Problem}!MaxDeviates@{MaxDeviates}}
\index{MaxDeviates@{MaxDeviates}!Problem@{Problem}}
\subsubsection{\setlength{\rightskip}{0pt plus 5cm}{\bf MSLVector} Problem::Max\-Deviates}\label{classProblem_m9}


Maximum displacement of geometry with respect to change in each variable.

\index{Problem@{Problem}!NumBodies@{NumBodies}}
\index{NumBodies@{NumBodies}!Problem@{Problem}}
\subsubsection{\setlength{\rightskip}{0pt plus 5cm}int Problem::Num\-Bodies}\label{classProblem_m1}


The number of rigid bodies.

\index{Problem@{Problem}!StateDim@{StateDim}}
\index{StateDim@{StateDim}!Problem@{Problem}}
\subsubsection{\setlength{\rightskip}{0pt plus 5cm}int Problem::State\-Dim}\label{classProblem_m2}


The dimenson of the state space.

\index{Problem@{Problem}!UpperState@{UpperState}}
\index{UpperState@{UpperState}!Problem@{Problem}}
\subsubsection{\setlength{\rightskip}{0pt plus 5cm}{\bf MSLVector} Problem::Upper\-State}\label{classProblem_m6}


{\bf MSLVector} {\rm (p.\,\pageref{classMSLVector})} of maximum values for each state variable.



The documentation for this class was generated from the following files:\begin{CompactItemize}
\item 
{\bf problem.h}\item 
{\bf problem.C}\end{CompactItemize}
