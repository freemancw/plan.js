\section{Multi\-Array  Class Template Reference}
\label{classMultiArray}\index{MultiArray@{Multi\-Array}}
A multidimensional array made from a 1D vector. 


{\tt \#include $<$marray.h$>$}

\subsection*{Public Methods}
\begin{CompactItemize}
\item 
{\bf Multi\-Array} (const vector$<$ int $>$ \&dims, const E \&{\bf x})
\begin{CompactList}\small\item\em Constructor with default assignment of x to each element.\item\end{CompactList}\item 
{\bf Multi\-Array} (const vector$<$ int $>$ \&dims)
\begin{CompactList}\small\item\em Constructor with no default assignment.\item\end{CompactList}\item 
{\bf Multi\-Array} ()
\begin{CompactList}\small\item\em Constructor with no initialization.\item\end{CompactList}\item 
{\bf $\sim$Multi\-Array} ()
\item 
E \& {\bf operator[$\,$]} (const vector$<$ int $>$ \&indices)
\begin{CompactList}\small\item\em This can be used for access or assignment (e.g., ma[indices] = 1).\item\end{CompactList}\item 
bool {\bf Increment} (vector$<$ int $>$ \&indices)
\begin{CompactList}\small\item\em Get the next element (used as an iterator). Return true if at end.\item\end{CompactList}\end{CompactItemize}
\subsection*{Public Attributes}
\begin{CompactItemize}
\item 
int {\bf Max\-Size}
\begin{CompactList}\small\item\em Maximum allowable array size (default = 10 million).\item\end{CompactList}\end{CompactItemize}
\subsection*{Friends}
\begin{CompactItemize}
\item 
istream \& {\bf operator$>$$>$} (istream \&is, Multi\-Array \&ma)
\begin{CompactList}\small\item\em This will not work correctly unless dimensions are preset correctly.\item\end{CompactList}\item 
ostream \& {\bf operator$<$$<$} (ostream \&os, const Multi\-Array \&ma)
\begin{CompactList}\small\item\em Just print out the vector.\item\end{CompactList}\end{CompactItemize}


\subsection{Detailed Description}
\subsubsection*{template$<$class E$>$ class Multi\-Array$<$ E $>$}

A multidimensional array made from a 1D vector.



\subsection{Constructor \& Destructor Documentation}
\index{MultiArray@{Multi\-Array}!MultiArray@{MultiArray}}
\index{MultiArray@{MultiArray}!MultiArray@{Multi\-Array}}
\subsubsection{\setlength{\rightskip}{0pt plus 5cm}template$<$class E$>$ Multi\-Array$<$ E $>$::Multi\-Array (const vector$<$ int $>$ \& {\em dims}, const E \& {\em x})}\label{classMultiArray_a0}


Constructor with default assignment of x to each element.

\index{MultiArray@{Multi\-Array}!MultiArray@{MultiArray}}
\index{MultiArray@{MultiArray}!MultiArray@{Multi\-Array}}
\subsubsection{\setlength{\rightskip}{0pt plus 5cm}template$<$class E$>$ Multi\-Array$<$ E $>$::Multi\-Array (const vector$<$ int $>$ \& {\em dims})}\label{classMultiArray_a1}


Constructor with no default assignment.

\index{MultiArray@{Multi\-Array}!MultiArray@{MultiArray}}
\index{MultiArray@{MultiArray}!MultiArray@{Multi\-Array}}
\subsubsection{\setlength{\rightskip}{0pt plus 5cm}template$<$class E$>$ Multi\-Array$<$ E $>$::Multi\-Array ()\hspace{0.3cm}{\tt  [inline]}}\label{classMultiArray_a2}


Constructor with no initialization.

\index{MultiArray@{Multi\-Array}!~MultiArray@{$\sim$MultiArray}}
\index{~MultiArray@{$\sim$MultiArray}!MultiArray@{Multi\-Array}}
\subsubsection{\setlength{\rightskip}{0pt plus 5cm}template$<$class E$>$ Multi\-Array$<$ E $>$::$\sim$Multi\-Array ()\hspace{0.3cm}{\tt  [inline]}}\label{classMultiArray_a3}




\subsection{Member Function Documentation}
\index{MultiArray@{Multi\-Array}!Increment@{Increment}}
\index{Increment@{Increment}!MultiArray@{Multi\-Array}}
\subsubsection{\setlength{\rightskip}{0pt plus 5cm}template$<$class E$>$ bool Multi\-Array$<$ E $>$::Increment (vector$<$ int $>$ \& {\em indices})\hspace{0.3cm}{\tt  [inline]}}\label{classMultiArray_a5}


Get the next element (used as an iterator). Return true if at end.

\index{MultiArray@{Multi\-Array}!operator[]@{operator[]}}
\index{operator[]@{operator[]}!MultiArray@{Multi\-Array}}
\subsubsection{\setlength{\rightskip}{0pt plus 5cm}template$<$class E$>$ E \& Multi\-Array$<$ E $>$::operator[$\,$] (const vector$<$ int $>$ \& {\em indices})\hspace{0.3cm}{\tt  [inline]}}\label{classMultiArray_a4}


This can be used for access or assignment (e.g., ma[indices] = 1).



\subsection{Friends And Related Function Documentation}
\index{MultiArray@{Multi\-Array}!operator<<@{operator$<$$<$}}
\index{operator<<@{operator$<$$<$}!MultiArray@{Multi\-Array}}
\subsubsection{\setlength{\rightskip}{0pt plus 5cm}template$<$class E$>$ ostream\& operator$<$$<$ (ostream \& {\em os}, const Multi\-Array$<$ E $>$ \& {\em ma})\hspace{0.3cm}{\tt  [friend]}}\label{classMultiArray_l1}


Just print out the vector.

\index{MultiArray@{Multi\-Array}!operator>>@{operator$>$$>$}}
\index{operator>>@{operator$>$$>$}!MultiArray@{Multi\-Array}}
\subsubsection{\setlength{\rightskip}{0pt plus 5cm}template$<$class E$>$ istream\& operator$>$$>$ (istream \& {\em is}, Multi\-Array$<$ E $>$ \& {\em ma})\hspace{0.3cm}{\tt  [friend]}}\label{classMultiArray_l0}


This will not work correctly unless dimensions are preset correctly.



\subsection{Member Data Documentation}
\index{MultiArray@{Multi\-Array}!MaxSize@{MaxSize}}
\index{MaxSize@{MaxSize}!MultiArray@{Multi\-Array}}
\subsubsection{\setlength{\rightskip}{0pt plus 5cm}template$<$class E$>$ int Multi\-Array::Max\-Size}\label{classMultiArray_m0}


Maximum allowable array size (default = 10 million).



The documentation for this class was generated from the following files:\begin{CompactItemize}
\item 
{\bf marray.h}\item 
{\bf marray.C}\end{CompactItemize}
